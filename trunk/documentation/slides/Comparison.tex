\documentclass[serif,9pt,xcolor=dvipsnames]{beamer}
\usetheme{default}

%\usepackage[english]{babel}    % deutsche Sonderzeichen, Trennmuster, etc
\usepackage[german]{babel}    % deutsche Sonderzeichen, Trennmuster, etc

\usepackage[T1]{fontenc} 
\usepackage[utf8]{inputenc}  % deutsche Umlaute


\usetheme[]{Warsaw}
%\usetheme[]{Rochester}

%  \usecolortheme[named=Brown]{structure} 
%\usecolortheme[]{crane} 
%\usecolortheme[]{rose} 
\usecolortheme[]{christian1} 

% Mathekram
%\usepackage{amsmath}
%\usepackage{amsfonts}
%\usepackage{amssymb}

% Graphiken
%\usepackage{epsfig}            % Paket zum Einbinden von postscript Graphiken
\usepackage{graphicx}		% Andere Graphikdateien einbinden
\usepackage{float}             % Hilfesmakros beim Positionieren von Graphiken
\usepackage{color}
%\usepackage{subfig}
\usepackage{textcomp}

\usepackage[sc]{mathpazo}	%FONT

\usepackage{relsize} % Notwenfig für verbatim
\usepackage{listings}
\usepackage{verbatim}


%% GLEGRAPHICS
\graphicspath{{./bilder/gleplots/}}

\newcommand{\includeglegraphics}[1]{  
%\includegraphics[]{#1.pdf}
\input{#1.inc}   
}

% \newcommand{\listing}[1]{
%    {\small 
%   \begin{lstlisting}
%    
%   \end{lstlisting}}
% }

\newcommand{\Th}{\vartheta}

%\newcommand{\Th}{\vartheta}


\definecolor{feedforwardpath}{rgb}{0.0,1,0.0}
\definecolor{feedbackpath}{rgb}{1.0,0.0,0.0}

\DeclareRobustCommand{\vec}[1]{{\mbox{\mathversion{bold}\ensuremath{#1}}}}
\DeclareRobustCommand{\mat}[1]{{\mbox{\mathversion{bold}\ensuremath{#1}}}}

\title[]{ORTD --- The Real-time Dynamics Framework}
	\subtitle{} %Mehrgrößenregelung einer Neuro-Prothese zur Generierung von Bewegungen  der oberen Extremität mittels Neuromuskulärer Elektrostimulation (NMES)
\date{May 2012}
\author{Christian Klauer$^{1}$\\
\tiny $^{1}$Control Systems Group, Technische Universität Berlin\\
Kontakt: klauer@control.tu-berlin.de
}




\begin{document}
% \frame[plain]{\titlepage}

% \author{Christian Klauer$^{1}$}





\begin{frame}
  \frametitle{Comparison to other systems - I}

{\small

  \begin{tabular}{p{0.18\linewidth}|p{0.24\linewidth}|p{0.23\linewidth}|p{0.23\linewidth}|}
  Feature & Xcos & Simulink & OpenRTDynamics \\
\hline
	Continuous-time simulation  &   Yes   &   Yes       &  only by embedding into Xcos (planned Simulink)         \\
\hline
	Realisation of Real-time Programms  &  code generation    &   code generation       &       real-time capable interpreter       \\
\hline
	Description language &  GUI-based Blocks  & GUI-based Blocks  &  textural description using Scilab    \\
\hline
	Ability to obtain well structured code &  superblocks  & subsystems,  &  superblocks for sharing code, conditional definition (like compiler flags), for loops, ..., allows to build powerful macros  \\
\hline
	Possible Target Systems &  Rtai, Real-time preemption, with much effort more & Nearly any target  &  Linux RT-Preemption and normal Linux: PC-based Linux, Android, embedded ARM devices  \\
  \end{tabular}

}

\end{frame}



\begin{frame}
  \frametitle{Comparison to other systems -- II}

{\small

  \begin{tabular}{p{0.18\linewidth}|p{0.24\linewidth}|p{0.23\linewidth}|p{0.23\linewidth}|}
  Feature & Xcos & Simulink & OpenRTDynamics \\
\hline
	Communication with RT-Programms  &  only if Rtai is used   &   Yes     &       Yes      \\
\hline
	Implementation of Logic structures &  using events (red lines) and signals; cumbersome and much effort  &   Stateflow, separation of continuous parts and logic    &     state machines that allow to combine continuous parts and logic; low implementation effort; start/stop/reset of superblocks      \\
\hline
	Multiple Threads with a separate main loop in each &  No  &   allows to start ``Tasks''  &       Yes: shared memory, ring buffer, events for communication      \\
\hline
	Time basis of the main loop &  regular timer &  regular timer  &  irregular time intervals possible, synchronisation to events e.g.: sensor data available, network packets arrived          \\
  \end{tabular}

}

\end{frame}


\begin{frame}
  \frametitle{Comparison to other systems -- III}

{\small

  \begin{tabular}{p{0.18\linewidth}|p{0.24\linewidth}|p{0.23\linewidth}|p{0.23\linewidth}|}
  Feature & Xcos & Simulink & OpenRTDynamics \\
\hline
	Including Scilab/Matlab Code into the RT-Program &  No  &  Code based a subset of the Matlab language can be compiled to C-Code  &  Yes: embedded Scilab interpreter, however no-deterministic execution time. Suggested to only use in separated threads  \\
\hline
	Replacing Code portions of the RT-Program while it is running &  No  &  No & Yes, the code of specially marked superblocks can be exchanged. The embedded Scilab interpreter can online-compile new superblocks \\
\hline
	Automating laboratory experiments with less effort &  No  &  Not sure & Yes, macros that greatly simplify automation tasks using Scilab-Code and replaceable superblocks are available \\

  \end{tabular}

}

\end{frame}


\end{document}
