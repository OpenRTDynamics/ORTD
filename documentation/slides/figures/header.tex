% \documentclass[%
% 	pdftex,%              PDFTex verwenden da wir ausschliesslich ein PDF erzeugen.
% 	a4paper,%             Wir verwenden A4 Papier.
% 	oneside,%             Einseitiger Druck.
% 	12pt,%                Grosse Schrift, besser geeignet für A4.
% 	halfparskip,%         Halbe Zeile Abstand zwischen Absätzen.
% 	%chapterprefix,%       Kapitel mit 'Kapitel' anschreiben.
% 	headsepline,%         Linie nach Kopfzeile.
% 	footsepline,%         Linie vor Fusszeile.
% 	bibtotocnumbered,%    Literaturverzeichnis im Inhaltsverzeichnis nummeriert einfügen.
% 	idxtotoc%             Index ins Inhaltsverzeichnis einfügen.
% ]{scrartcl}
% 
% 
% 
% 

% %\usepackage{a4}                % Paket fuer A4 Seitenformat
% %\usepackage{epsfig}            % Paket zum Einbinden von postscript Graphiken
% %\usepackage[ngerman]{babel}    % deutsche Sonderzeichen, Trennmuster, etc
% %\usepackage{german}
% \usepackage[T1]{fontenc} 
% \usepackage[utf8]{inputenc}  % deutsche Umlaute
% \usepackage{float}             % Hilfesmakros beim Positionieren von Graphiken
% \usepackage{amsmath}
% \usepackage{amsfonts}
% \usepackage{amssymb}
% \usepackage{graphicx}		% Andere Graphikdateien einbinden
% 
% %\usepackage{supertabular}
% 
% \usepackage[sc]{mathpazo}

%\usepackage{tikz}

%\usepackage{rawfonts}
%\usepackage{pictex}

%\usepackage{epic}


%
% 13a. Font 'Latin Modern Family' verwenden.
%      Verwende dieses Paket wenn du DML selbst kompilierst.
%
%\usepackage{lmodern}

%
% 14. Typewriter Font LuxiMono laden.
%
%\usepackage[scaled=.85]{luximono}

% 
% 
% %% Zeilenabstand: 1 1/2-zeilig   (1.3)
% \renewcommand{\baselinestretch}{1.1}
% 
% %% Absatzabstand
% \parskip1.5ex
% 
% %% erste Zeile eines Absatzes nicht einrcken
% \parindent0em
% 
% %% Seitenumbruchsteuerung, bei wenigen Zeilen nach Überschrift am Zeilenende
% \def\condbreak#1{\vskip 0pt plus #1\pagebreak[3]\vskip 0pt plus -#1\relax}
% 
% 
% \setlength{\textwidth}{16.5cm}
% \setlength{\oddsidemargin}{0mm}








  \documentclass{article}


  \usepackage{german,amssymb,amsmath}


  \renewcommand{\familydefault}{\sfdefault}


% %\usepackage{a4}                % Paket fuer A4 Seitenformat
% %\usepackage{epsfig}            % Paket zum Einbinden von postscript Graphiken
% %\usepackage[ngerman]{babel}    % deutsche Sonderzeichen, Trennmuster, etc
% %\usepackage{german}
% \usepackage[T1]{fontenc} 
% \usepackage[utf8]{inputenc}  % deutsche Umlaute
% \usepackage{float}             % Hilfesmakros beim Positionieren von Graphiken
 \usepackage{amsmath}
 \usepackage{amsfonts}
 \usepackage{amssymb}
% \usepackage{graphicx}		% Andere Graphikdateien einbinden
% 
% %\usepackage{supertabular}
% 
 \usepackage[sc]{mathpazo}

\newcommand{\be}{\begin{equation*}}
\newcommand{\ee}{\end{equation*}}
\newcommand{\bea}{\begin{ray*}}
\newcommand{\eea}{\end{eqnarray*}}
\newcommand{\R}{\mathbb{R}}
\newcommand{\pd}{d}
\newcommand{\intd}{\text{d}}
\newcommand{\Lap}{\mathcal{L}}
\newcommand{\lap}{\;\;\laplace\;\;}
\newcommand{\jw}{j\omega}
\newcommand{\w}{\omega}
\newcommand{\phii}{\varphi}
\newcommand{\gdw}{\Longleftrightarrow}
\newcommand{\vect}[1]{\boldsymbol{#1}}
\newcommand{\D}{\displaystyle}
\newcommand{\NN}{\nonumber}

\DeclareRobustCommand{\vec}[1]{{\mbox{\mathversion{bold}\ensuremath{#1}}}}
\DeclareRobustCommand{\mat}[1]{{\mbox{\mathversion{bold}\ensuremath{#1}}}}
\newcommand{\FSET}{\protect\mathscr}
\newcommand{\OP}{\protect\mathcal}
\newcommand{\POLY}{\protect\mathsf}
% --- Definition von Mengensymbolen ----
\providecommand{\Rset}{\protect\mathbb{R}}
\providecommand{\Nset}{\protect\mathbb{N}}
\providecommand{\Zset}{\protect\mathbb{Z}}
\providecommand{\Qset}{\protect\mathbb{Q}}
% --- Konstante e=2.71828 soll nicht kursiv erscheinen ---
\newcommand{\e}{\mathrm{e}}
% --- Komplexe Einheit j ---
\newcommand{\jc}{\dot{\imath}}
% --- Differentialoperator senkrecht setzen ---
\renewcommand{\d}{\operatorname{d}}

\newcommand{\vectornorm}[1]{\left|\left|#1\right|\right|}

\newcommand{\Th}{\vartheta}
\renewcommand{\Theta}{\vartheta}




